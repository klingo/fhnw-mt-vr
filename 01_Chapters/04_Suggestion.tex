%----------------------------------------------------------------------------------------
%	CHAPTER - SUGGESTION
%----------------------------------------------------------------------------------------

\chapter{Suggestion}

\label{ChapterSuggestion}

%----------------------------------------------------------------------------------------
%	SECTION 1
%----------------------------------------------------------------------------------------

\section{Introduction}

The in chapter \ref{DSRCycle} discussed \gls{dsr} cycle from \cite{Vaishnavi2008} and \cite{Hevner2010} defines \textit{Suggestion} as the next step after \textit{Awareness of Problem}. \cite{Vaishnavi2008} define the \textit{Suggestion} phase as a creative step in which new functionality is envisioned to form a tentative design for the development of the prototype. \cite{Vaishnavi2008} further specifies the output as a constructed theory that addresses the identified problem, which can include new ideas and concepts, new methods, and new models that in the end shall be validated with the development of the prototype. \newline
In a first step, the end goals of the design are defined. Then the individual views of presentation (e.g. an overview and detail view) are defined as well as the navigation between them. As the third step, the visualisation of the individual views are defined before finally the interaction patterns (action/reaction) can be mapped to them.


%----------------------------------------------------------------------------------------
%	SECTION 2
%----------------------------------------------------------------------------------------

\section{Goals of the Design}

In order to define the goals of the design, both the specific characteristics of \gls{vr} itself and the to be visualised data have to be considered. In this thesis, the goals are defined either as \glspl{mdg} for the more overarching goals, or as \glspl{sdg} which are focussing rather on specific areas. In the following two sub-chapters, these characteristics are discussed in more detail and the individual goals are defined.


%-----------------------------------
%	SUBSECTION 1
%-----------------------------------
\subsection{\gls{vr}-specific Goals}

The goals that are derived from \gls{vr} itself, are more high-level since they can be applied to any \gls{vr} application independent of what its purpose is. Therefore, the three important characteristics of \gls{vr} as defined by \cite{Stone1994} and discussed in chapter \ref{SubSectionVisualisationManipulation} are considered as an integral part of the design. These characteristics can be summarized in the following three keywords: Action/Reaction, Immersion and Spatial. Based on these, the first \glspl{mdg} from a \gls{vr} perspective can be derived and further described with examples:
\begin{itemize}[noitemsep,nolistsep]
	\item \textbf{\gls{mdg} 1:} High interactivity with tightly coupled actions/reactions \newline
		\textit{Example: No slow animations, no timed events, or any other change in scenery without a trigger from the user}
	\item \textbf{\gls{mdg} 2:} The user is part of the \gls{ve} and thus is able to 'travel' around the visualisations \newline
		\textit{Example: Movement at free will, and no outside view, or third-person \gls{pov}}
	\item \textbf{\gls{mdg} 3:} The design is intended for \gls{vr} and thus in full 3D \newline
		\textit{Example: No flattened two-dimensional representation in 3D space}
\end{itemize}
In addition to the above characteristics, another goal has to be that the design should not rely on specific \gls{vr}-hardware, but rather define them in a more abstract way as to what the individual hardware items should be capable of. Since this topic is only important for the \gls{vr} input devices and potentially the \gls{hmd} but not the visualisation itself, it is considered as a \gls{sdg}.
\begin{itemize}[noitemsep,nolistsep]
	\item \textbf{\gls{sdg} 1:} The design is independent of specific \gls{vr} hardware \newline
	\textit{Example: No direct references to specific hardware, but rather what the hardware has to be capable of (requirements)}
\end{itemize}
From a \gls{vr} perspective, the important areas can all be covered with the above defined goals. The main aspects of the design are depending on what data is used and thus is discussed in the following sub-chapter.


%-----------------------------------
%	SUBSECTION 2
%-----------------------------------

\subsection{Data-specific Goals}

The data set that is used as the 'categorized financial data' is shown in Appendix \ref{AppendixA}. It consists of eight columns that contain information about transactions, such as the date, the currency and amount, the recipient and the categorizing columns: \textit{Main category} and \textit{Subcategory}. A complete overview and mapping of these is shown below in Table \ref{tbl:financialcategories}.

\begin{longtable}{ | p{5cm} | p{9cm} |}
	\hline
		\textbf{Main category} & \textbf{Subcategory} \\
	\hline
	\endfirsthead % Line(s) to appear as head of the table on the first page
	\multicolumn{2}{c}%
		{\tablename\ \thetable\ -- \textit{Continued from previous page}} \\
	\hline
		\textbf{Main category} & \textbf{Subcategory} \\
	\hline
	\endhead % Line(s) to appear at top of every page (except first)
	\hline
		\multicolumn{2}{r}{\textit{Continued on next page}} \\
	\endfoot % Last line(s) to appear at the bottom of every page (except last)
	\endlastfoot % Last line(s) to appear at the end of the table
	\hline
	Communication \& media &
	- Film, photo, electronic devices and accessories \newline
	- Miscellaneous \newline
	- Multimedia (music, video \& apps) \newline
	- Newspaper and magazine subscriptions \newline
	- Radio and television fees \newline
	- Software \newline
	- Telephone,  Internet and TV \\
	\hline
	Health &
	- Medical services \\
	\hline
	Household &
	- Children and family \newline
	- Food and beverage \newline
	- Household articles and accessories \newline
	- Household equipment \newline
	- Office articles and services \newline
	- Pets \\
	\hline
	Income \& credits &
	- Capital revenues (interest, dividends \& earnings) \newline
	- Gifts and inheritance \newline
	- Refunds \newline
	- Salary and sideline \newline
	- Sale of property \\
	\hline
	Leisure time, sport \& hobby &
	- Books and literature \newline
	- Going out, culture and cinema \newline
	- Miscellaneous \newline
	- Toys and hobby articles \\
	\hline
	Living \& energy &
	- Building and property insurance \newline
	- Electricity and gas \newline
	- Rent and mortgage interest \newline
	- Tools and garden \\
	\hline
	Other expenses &
	- Banking services and charges \newline
	- Benefactor contributions \newline
	- Credit card invoice and fees \newline
	- Loan and debt interest \newline
	- Miscellaneous \newline
	- Repayments \\
	\hline
	Personal expenditure &
	- Clothing, shoes and accessories \newline
	- Donations \newline
	- Food (snacks, restaurants and bars) \newline
	- Gifts \newline
	- Miscellaneous \newline
	- Personal hygiene and wellness \newline
	- Training and further education \\
	\hline
	Taxes \& duties &
	- Community and cantonal tax \newline
	- Federal tax \newline
	- Fees \newline
	- Military exemption tax \\
	\hline
	Traffic, car \& transport &
	- Fuel (gasoline, diesel, gas) \newline
	- Public transport (tickets \& subscriptions) \newline
	- Traffic charges \\
	\hline
	Vacation \& travel &
	- Accommodation and hotels \newline
	- Miscellaneous \newline
	- Offers and services \newline
	- Travel and flight costs \\
	\hline
	Withdrawals &
	- Bancomat \newline
	- Teller (branch) \\
	\hline
	\caption{Mapping of Main category and Subcategory of financial data set}
	\label{tbl:financialcategories}
\end{longtable}
	
Based on this, ...


Furthermore, relating to the \gls{vism} which is discussed in chapter \ref{SubSectionVISM}, 
mantra: Overview first, zoom and filter, then details-on-demand

They further conclude that due to the different goals for the visualization (e.g. testing a theory or understanding the data set), it makes it difficult on a general level to discuss the problems with how the visualization should be supported \citep{Stone1994}.


- Goal of the Design (multi-month view, compare data, visual relative evaluation (colour)

As a high level guideline for the \textit{Suggestion}, the three important characteristics of VR, defined by \cite{Stone1994} and discussed in chapter \ref{SubSectionVisualisationManipulation}, are seen as an integral part of the design. They can be summarized in: Action/Reaction, Immersion, and Spatial. \newline




VR allows us to create Virtual Environments (VEs) in which we can render our 3D objects representing our data. The advantage that these VEs have over traditional approaches is that they allows us to be immersed within the data. We can use methods to examine the different features of the data that are more intuitive to us. An example of this is the ability to track the users head position so that we can appear to look around object, this is how as humans we are familiar with examining objects of interest, rather than moving a mouse. In a totally immersive virtual environment we can use body movement to walk around objects or put our head inside virtual representation of our data. Also within an immersive environment it is possible to map the users hand position in the real world to a virtual hand in the VE, therefore allowing the user to manipulate virtual objects.
\cite{Jamieson2007}
--> IMMERSION
--> HEAD TRACKING
--> BODY MOVEMENT
--> ACCURATE HAND TRACKING
--> DATA MANIPULATION BY HAND




%----------------------------------------------------------------------------------------
%	SECTION 3
%----------------------------------------------------------------------------------------

\section{View Definition and Navigation Map}


- Navigation Map / "Screen-Flow"


%----------------------------------------------------------------------------------------
%	SECTION 4
%----------------------------------------------------------------------------------------

\section{Visualisation of Views and Data}

- Exact Presentation of Data (how look like?)


%----------------------------------------------------------------------------------------
%	SECTION 5
%----------------------------------------------------------------------------------------

\section{Mapping of Interaction Patterns}

- Action/Reaction (Interaction)




%----------------------------------------------------------------------------------------
%	SECTION 6
%----------------------------------------------------------------------------------------

\section{Conclusion}



%% IDEA FOR PROTOTYPE

- Start with a small table on which a house etc is visualized. floating above is year (+month)
- Each object represents one category from the financial expenses.
- maybe size indicates the overall amount?
- The colours depends on the difference between my planned expenses (or average expenses) and the actual expenses. less = green, about the same = yellow-(green-)ish, slightly above = orange, above = red.
- By clicking on one of the objects, it gets highlighted and a line-chart appears showing the expenses
   A) show individual transactions for the given month up until the threshold
   B) show individual months until the threshold (+forecast?)
- show multiple lines for the different sub-categories (enable/disable) plus the total
- clicking on an entry of the line chart displays details about transaction (amount, location), or the month (amount) in an overlay
- switching between single-month and year view by... using the touchpad (up/down clicks)
- navigating through months/years by... a using the touchpad! (left/right clicks)
- resetting the view to start by... clicking on the select button

- maybe outside as rotating rings: the individual bank accounts?

