%----------------------------------------------------------------------------------------
%	CHAPTER - SUGGESTION
%----------------------------------------------------------------------------------------

\chapter{Suggestion} % Main chapter title

\label{ChapterSuggestion} % Change X to a consecutive number; for referencing this chapter elsewhere, use \ref{ChapterX}

%----------------------------------------------------------------------------------------
%	SECTION 1
%----------------------------------------------------------------------------------------

\section{Introduction}

Suggestion. The Suggestion phase follows immediately behind the Proposal and is intimately connected with it, as the dotted line around Proposal and Tentative Design (the output of the Suggestion phase) indicates. Indeed, in any formal proposal for design science research, such as one to be made to the NSF (National Science Foundation) or an industry sponsor, a Tentative Design and likely the performance of a prototype based on that design would be an integral part of the Proposal. Moreover, if after consideration of an interesting problem, a Tentative Design does not present itself to the researcher, the idea (Proposal) will be set aside. Suggestion is an essentially creative step wherein new functionality is envisioned based on a novel configuration of either existing or new and existing elements. The step has been criticized as introducing nonrepeatability into the design science research method; human creativity is still a poorly understood cognitive process. However, the step has necessary analogues in all research methods; for example, in positivist research, creativity is inherent in the leap from curiosity about organizational phenomena to the development of appropriate constructs that operationalize the phenomena and an appropriate research design for their measurement.
\cite{Vaishnavi2008}

3. Construct a theory that addresses the problem. A theory is a set of propositions that identifies units, states of units, and laws or beliefs about the interaction of units to explain, predict, and describe observations within some boundary. It includes new ideas and concepts, conceptual frameworks, new methods, and models (e.g., mathematical models, simulation models, and data models). Direct the prototype design and development effort to validate or invalidate the theory.
\cite{Vaishnavi2008}

\cite{Stone1994} further continue to define three important characteristics that \gls{vr} has in this regard:
\begin{itemize}[noitemsep,nolistsep]
	\item \gls{vr} exhibits high interactivity (the user's actions and the caused reactions are tightly coupled together)
	\item \gls{vr} support embodiment (the user is represented  in the same spatial framework as the data)
	\item The \gls{vr} representation is spatial in nature (all virtual objects are placed in a spatial framework)
\end{itemize}
--> IMMERSION
--> SPATIAL
--> ACTION/REACTION

VR allows us to create Virtual Environments (VEs) in which we can render our 3D objects representing our data. The advantage that these VEs have over traditional approaches is that they allows us to be immersed within the data. We can use methods to examine the different features of the data that are more intuitive to us. An example of this is the ability to track the users head position so that we can appear to look around object, this is how as humans we are familiar with examining objects of interest, rather than moving a mouse. In a totally immersive virtual environment we can use body movement to walk around objects or put our head inside virtual representation of our data. Also within an immersive environment it is possible to map the users hand position in the real world to a virtual hand in the VE, therefore allowing the user to manipulate virtual objects.
\cite{Jamieson2007}
--> IMMERSION
--> HEAD TRACKING
--> BODY MOVEMENT
--> ACCURATE HAND TRACKING
--> DATA MANIPULATION BY HAND


%% IDEA FOR PROTOTYPE

- Start with a small table on which a house etc is visualized. floating above is year (+month)
- Each object represents one category from the financial expenses.
- maybe size indicates the overall amount?
- The colours depends on the difference between my planned expenses (or average expenses) and the actual expenses. less = green, about the same = yellow-(green-)ish, slightly above = orange, above = red.
- By clicking on one of the objects, it gets highlighted and a line-chart appears showing the expenses
   A) show individual transactions for the given month up until the threshold
   B) show individual months until the threshold (+forecast?)
- show multiple lines for the different sub-categories (enable/disable) plus the total
- clicking on an entry of the line chart displays details about transaction (amount, location), or the month (amount) in an overlay
- switching between single-month and year view by... using the touchpad (up/down clicks)
- navigating through months/years by... a using the touchpad! (left/right clicks)
- resetting the view to start by... clicking on the select button

- maybe outside as rotating rings: the individual bank accounts?




%-----------------------------------
%	SUBSECTION 1
%-----------------------------------
\subsection{Subsection 1}

tbd


%-----------------------------------
%	SUBSECTION 2
%-----------------------------------

\subsection{Subsection 2}

tbd


%----------------------------------------------------------------------------------------
%	SECTION 2
%----------------------------------------------------------------------------------------

\section{Main Section 2}

tbd