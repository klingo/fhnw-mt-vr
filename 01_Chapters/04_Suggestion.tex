%----------------------------------------------------------------------------------------
%	CHAPTER - SUGGESTION
%----------------------------------------------------------------------------------------

\chapter{Suggestion}

\label{ChapterSuggestion}

%----------------------------------------------------------------------------------------
%	SECTION 1
%----------------------------------------------------------------------------------------

\section{Introduction}

The in chapter \ref{DSRCycle} discussed \gls{dsr} cycle from \cite{Vaishnavi2008} and \cite{Hevner2010} defines \textit{Suggestion} as the next step after \textit{Awareness of Problem}. \newline
\cite{Vaishnavi2008} define the \textit{Suggestion} phase as a creative step in which new functionality is envisioned to form a tentative design for the development of the prototype. \cite{Vaishnavi2008} further specifies the output as a constructed theory that addresses the identified problem, which can include new ideas and concepts, new methods, and new models that in the end shall be validated with the development of the prototype. \newline
In a first step, the end goals of the design are defined. Then the individual states of presentation (e.g. an overview and detail view) are defined as well as the navigation between them. As the third step, the visualisation of the individual states are defined before finally the interaction patterns (action/reaction) can be mapped to them.


%----------------------------------------------------------------------------------------
%	SECTION 2
%----------------------------------------------------------------------------------------

\section{Goals of the Design}

In order to define the goals of the Design, both the specific characteristics of \gls{vr} itself and the to be visualised data has to be considered. While the \gls{vr}-specific


In order to define the \glspl{sdg}, the characteristics of the data-set that ultimately should be visualised has to be looked at.




%-----------------------------------
%	SUBSECTION 1
%-----------------------------------
\subsection{\gls{vr}-specific Goals}

As a high level guideline for the Design, the three important characteristics of VR, as defined by \cite{Stone1994} and discussed in chapter \ref{SubSectionVisualisationManipulation}, are seen as an integral part of the design. These characteristics can be summarized in: Action/Reaction, Immersion and Spatial. Based on this, the first \glspl{mdg} can be derived:
\begin{itemize}[noitemsep,nolistsep]
	\item \gls{mdg} 1: High interactivity with tightly coupled actions/reactions (i.e. no slow animations, no timed events, or any other change in scenery without a trigger from the user)
	\item \gls{mdg} 2: The user is part of the \gls{ve} and thus is able to 'travel' around the visualisations (i.e. no outside view, or third-person \gls{pov})
	\item \gls{mdg} 3: The design is intended for \gls{vr} and thus in full 3D and not just a two-dimensional representation
\end{itemize}




%-----------------------------------
%	SUBSECTION 2
%-----------------------------------

\subsection{Data-specific Goals}

tbd

- Goal of the Design (multi-month view, compare data, visual relative evaluation (colour)

As a high level guideline for the \textit{Suggestion}, the three important characteristics of VR, defined by \cite{Stone1994} and discussed in chapter \ref{SubSectionVisualisationManipulation}, are seen as an integral part of the design. They can be summarized in: Action/Reaction, Immersion, and Spatial. \newline




VR allows us to create Virtual Environments (VEs) in which we can render our 3D objects representing our data. The advantage that these VEs have over traditional approaches is that they allows us to be immersed within the data. We can use methods to examine the different features of the data that are more intuitive to us. An example of this is the ability to track the users head position so that we can appear to look around object, this is how as humans we are familiar with examining objects of interest, rather than moving a mouse. In a totally immersive virtual environment we can use body movement to walk around objects or put our head inside virtual representation of our data. Also within an immersive environment it is possible to map the users hand position in the real world to a virtual hand in the VE, therefore allowing the user to manipulate virtual objects.
\cite{Jamieson2007}
--> IMMERSION
--> HEAD TRACKING
--> BODY MOVEMENT
--> ACCURATE HAND TRACKING
--> DATA MANIPULATION BY HAND




%----------------------------------------------------------------------------------------
%	SECTION 3
%----------------------------------------------------------------------------------------

\section{State Definition and Navigation Map}


- Navigation Map / "Screen-Flow"


%----------------------------------------------------------------------------------------
%	SECTION 4
%----------------------------------------------------------------------------------------

\section{Visualisation of States and Data}

- Exact Presentation of Data (how look like?)


%----------------------------------------------------------------------------------------
%	SECTION 5
%----------------------------------------------------------------------------------------

\section{Mapping of Interaction Patterns}

- Action/Reaction (Interaction)




%----------------------------------------------------------------------------------------
%	SECTION 6
%----------------------------------------------------------------------------------------

\section{Conclusion}



%% IDEA FOR PROTOTYPE

- Start with a small table on which a house etc is visualized. floating above is year (+month)
- Each object represents one category from the financial expenses.
- maybe size indicates the overall amount?
- The colours depends on the difference between my planned expenses (or average expenses) and the actual expenses. less = green, about the same = yellow-(green-)ish, slightly above = orange, above = red.
- By clicking on one of the objects, it gets highlighted and a line-chart appears showing the expenses
   A) show individual transactions for the given month up until the threshold
   B) show individual months until the threshold (+forecast?)
- show multiple lines for the different sub-categories (enable/disable) plus the total
- clicking on an entry of the line chart displays details about transaction (amount, location), or the month (amount) in an overlay
- switching between single-month and year view by... using the touchpad (up/down clicks)
- navigating through months/years by... a using the touchpad! (left/right clicks)
- resetting the view to start by... clicking on the select button

- maybe outside as rotating rings: the individual bank accounts?

