%----------------------------------------------------------------------------------------
%	CHAPTER - SUGGESTION
%----------------------------------------------------------------------------------------

\chapter{Suggestion} % Main chapter title

\label{ChapterSuggestion} % Change X to a consecutive number; for referencing this chapter elsewhere, use \ref{ChapterX}

%----------------------------------------------------------------------------------------
%	SECTION 1
%----------------------------------------------------------------------------------------

\section{Main Section 1}

Suggestion. The Suggestion phase follows immediately behind the Proposal and is intimately connected with it, as the dotted line around Proposal and Tentative Design (the output of the Suggestion phase) indicates. Indeed, in any formal proposal for design science research, such as one to be made to the NSF (National Science Foundation) or an industry sponsor, a Tentative Design and likely the performance of a prototype based on that design would be an integral part of the Proposal. Moreover, if after consideration of an interesting problem, a Tentative Design does not present itself to the researcher, the idea (Proposal) will be set aside. Suggestion is an essentially creative step wherein new functionality is envisioned based on a novel configuration of either existing or new and existing elements. The step has been criticized as introducing nonrepeatability into the design science research method; human creativity is still a poorly understood cognitive process. However, the step has necessary analogues in all research methods; for example, in positivist research, creativity is inherent in the leap from curiosity about organizational phenomena to the development of appropriate constructs that operationalize the phenomena and an appropriate research design for their measurement.
\cite{Vaishnavi2008}


\cite{Stone1994} further continue to define three important characteristics that \gls{vr} has in this regard:
\begin{itemize}[noitemsep,nolistsep]
	\item \gls{vr} exhibits high interactivity (the user's actions and the caused reactions are tightly coupled together)
	\item \gls{vr} support embodiment (the user is represented  in the same spatial framework as the data)
	\item The \gls{vr} representation is spatial in nature (all virtual objects are placed in a spatial framework)
\end{itemize}
--> IMMERSION
--> SPATIAL
--> ACTION/REACTION

VR allows us to create Virtual Environments (VEs) in which we can render our 3D objects representing our data. The advantage that these VEs have over traditional approaches is that they allows us to be immersed within the data. We can use methods to examine the different features of the data that are more intuitive to us. An example of this is the ability to track the users head position so that we can appear to look around object, this is how as humans we are familiar with examining objects of interest, rather than moving a mouse. In a totally immersive virtual environment we can use body movement to walk around objects or put our head inside virtual representation of our data. Also within an immersive environment it is possible to map the users hand position in the real world to a virtual hand in the VE, therefore allowing the user to manipulate virtual objects.
\cite{Jamieson2007}
--> IMMERSION
--> HEAD TRACKING
--> BODY MOVEMENT
--> ACCURATE HAND TRACKING
--> DATA MANIPULATION BY HAND


%% IDEA FOR PROTOTYPE

- Start with a small table on which a house etc is visualized. floating above is year (+month)
- Each object represents one category from the financial expenses.
- maybe size indicates the overall amount?
- The colours depends on the difference between my planned expenses (or average expenses) and the actual expenses. less = green, about the same = yellow-(green-)ish, slightly above = orange, above = red.
- By clicking on one of the objects, it gets highlighted and a line-chart appears showing the expenses
   A) show individual transactions for the given month up until the threshold
   B) show individual months until the threshold (+forecast?)
- show multiple lines for the different sub-categories (enable/disable) plus the total
- clicking on an entry of the line chart displays details about transaction (amount, location), or the month (amount) in an overlay
- switching between single-month and year view by... using the touchpad (up/down clicks)
- navigating through months/years by... a using the touchpad! (left/right clicks)
- resetting the view to start by... clicking on the select button

- maybe outside as rotating rings: the individual bank accounts?




%-----------------------------------
%	SUBSECTION 1
%-----------------------------------
\subsection{Subsection 1}

Nunc posuere quam at lectus tristique eu ultrices augue venenatis. Vestibulum ante ipsum primis in faucibus orci luctus et ultrices posuere cubilia Curae; Aliquam erat volutpat. Vivamus sodales tortor eget quam adipiscing in vulputate ante ullamcorper. Sed eros ante, lacinia et sollicitudin et, aliquam sit amet augue. In hac habitasse platea dictumst.

%-----------------------------------
%	SUBSECTION 2
%-----------------------------------

\subsection{Subsection 2}
Morbi rutrum odio eget arcu adipiscing sodales. Aenean et purus a est pulvinar pellentesque. Cras in elit neque, quis varius elit. Phasellus fringilla, nibh eu tempus venenatis, dolor elit posuere quam, quis adipiscing urna leo nec orci. Sed nec nulla auctor odio aliquet consequat. Ut nec nulla in ante ullamcorper aliquam at sed dolor. Phasellus fermentum magna in augue gravida cursus. Cras sed pretium lorem. Pellentesque eget ornare odio. Proin accumsan, massa viverra cursus pharetra, ipsum nisi lobortis velit, a malesuada dolor lorem eu neque.

%----------------------------------------------------------------------------------------
%	SECTION 2
%----------------------------------------------------------------------------------------

\section{Main Section 2}

Sed ullamcorper quam eu nisl interdum at interdum enim egestas. Aliquam placerat justo sed lectus lobortis ut porta nisl porttitor. Vestibulum mi dolor, lacinia molestie gravida at, tempus vitae ligula. Donec eget quam sapien, in viverra eros. Donec pellentesque justo a massa fringilla non vestibulum metus vestibulum. Vestibulum in orci quis felis tempor lacinia. Vivamus ornare ultrices facilisis. Ut hendrerit volutpat vulputate. Morbi condimentum venenatis augue, id porta ipsum vulputate in. Curabitur luctus tempus justo. Vestibulum risus lectus, adipiscing nec condimentum quis, condimentum nec nisl. Aliquam dictum sagittis velit sed iaculis. Morbi tristique augue sit amet nulla pulvinar id facilisis ligula mollis. Nam elit libero, tincidunt ut aliquam at, molestie in quam. Aenean rhoncus vehicula hendrerit.