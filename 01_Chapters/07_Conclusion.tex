%----------------------------------------------------------------------------------------
%	CHAPTER - CONCLUSION
%----------------------------------------------------------------------------------------

\chapter{Conclusion}

\label{ChapterConclusion}

%----------------------------------------------------------------------------------------
%	SECTION 1
%----------------------------------------------------------------------------------------

\section{Introduction}

TODO: Answer to Research Questions? Create a new one for the main part?

Conclusion. This phase is the finale of a specific research effort. Typically, it is the result of satisficing; that is, although there are still deviations in the behavior of the artifact from the (multiply) revised hypothetical predictions, the results are adjudged “good enough.” Not only are the results of the effort consolidated and “written up” at this phase, but the knowledge gained in the effort is fre- quently categorized as either “firm” — facts that have been learned and can be repeatedly applied or behavior that can be repeatedly invoked — or as “loose ends” — anomalous behavior that defies explanation and may well serve as the subject of further research.
\cite{Vaishnavi2008}



%----------------------------------------------------------------------------------------
%	SECTION 2
%----------------------------------------------------------------------------------------

\section{Summary}

tbd



%----------------------------------------------------------------------------------------
%	SECTION 3
%----------------------------------------------------------------------------------------

\section{Findings}

tbd



%----------------------------------------------------------------------------------------
%	SECTION 4
%----------------------------------------------------------------------------------------

\section{Future Research}

tbd