%----------------------------------------------------------------------------------------
%	CHAPTER - CONCLUSION
%----------------------------------------------------------------------------------------

\chapter{Conclusion}

\label{ChapterConclusion}

The final chapter in this thesis covers the final phase of the \gls{dsr} cycle from \cite{Vaishnavi2008}, the \textit{Conclusion}. They classify this phase as the one where the results of the effort is consolidated and the knowledge gained categorized and documented \citep{Vaishnavi2008}.


%----------------------------------------------------------------------------------------
%	SECTION 1
%----------------------------------------------------------------------------------------

\section{Introduction}

In the first part of the conclusion, a summary of the research done in this thesis is provided. Then the main findings of the literature review and the design, implementation and evaluation of the prototype application are provided, as well as an answer to the \gls{mrq} from the first chapter. Finally, a suggestion for future research is given.


%----------------------------------------------------------------------------------------
%	SECTION 2
%----------------------------------------------------------------------------------------

\section{Summary}

Beginning with the problem statement that information about the personal financial situation is till presented in the same ways like a decade ago, and that only little research has been made in how to visualize data in \gls{vr}, this thesis proposed a solution approach to close this gap. With the first \gls{srq}, research about the different methods of user input in \gls{vr} was conducted, which showed their individual advantages and disadvantages compared with others. Next, with the second \gls{srq} the focus was on different interaction patterns of \gls{vr} where a classification into \textit{Travel}, \textit{Selection}, and \textit{Manipulation} can be found. The \gls{vism} give some more guidelines in what tasks should be available to the user for an effective information seeking. With the third \gls{srq}, different approaches for the visualisation of data in \gls{vr} have been analysed and a close relation to the type of data could be seen. Alongside the \textit{Suggestion} phase of the \gls{dsr} cycle where a design for a prototype application is proposed, a first part of \gls{srq} is discussed in terms of what the potential benefits can be by bringing the interaction with categorized financial data into \gls{vr}. Building up on that, the \textit{Development} and \textit{Evaluation} phase confirmed the technical feasibility of the proposed design and showed its effectiveness in a set of different scenarios. Since the prototype is much more limited in terms of filtering possibilities, the evaluation is heavily depending on the scenario. With the right focus, the prototype clearly beats the current e-banking solution in terms of effectiveness and comprehensibility. In other cases it is somewhat sub-par to the traditional e-banking solution but still is able to allow for a successful completion of the desired task.


%----------------------------------------------------------------------------------------
%	SECTION 3
%----------------------------------------------------------------------------------------

\section{Findings}

The following findings were made during the thesis:
\begin{itemize}[]
	\item noitemsep = off
	\item nolistsep = off
\end{itemize}


With respect to the thesis statement, it can be said that the level of enhancement is depending on the specific scenario of the exploratory analysis. Due to the multiple linked views, the prototype can show its strengths when the scenario asks for switches in these views, such as comparing one month with another which can be done with just one click in the prototype application whereas in e-banking the selected date-range has to be manually adjusted. Contrary to that, with a very detailed question the prototype comes soon to its limitations as in its current state it only supports a filtering for categories and full years/months/days, but not for individually defined date-ranges or any other criteria such as amount-ranges or account selection. This however does not mean that the answer cannot be found with the prototype as well, it is just not as effective to do so as with other scenarios.
\begin{framed}
	\textit{\thesisstatementtext}
\end{framed}

Based on these results, the thesis statement as defined above, is not rejected.


%----------------------------------------------------------------------------------------
%	SECTION 4
%----------------------------------------------------------------------------------------

\section{Future Research}

With regards to the different \glspl{srq}, future research in the following areas could lead to new knowledge for the academic community and help on further improving the prototype application.
\begin{itemize}[]
	\item \textbf{\gls{srq} 1:} As mentioned in the findings, the weaknesses of the prototype application are related to very specific queries that have to be executed. As an alternative to just provide more buttons and input fields in the \gls{ve} with the risk of overloading it, the utilization of \textit{Speech Recognition} integrated in the \gls{vr} application could potentially generate massive improvements. More advanced filtering such as "Show me all transactions over CHF 100 in the last seven days" would become possible.
	
	\item \textbf{\gls{srq} 2:} From the \gls{vism} 2.0, only the main tasks were considered for this thesis, whereas the three support tasks \textit{History}, \textit{Extract}, and \textit{Collaboration} were not in focus. Especially the \textit{Collaboration} support task could allow for even better exploratory analysis when the \gls{ve} is shared with additional people. In terms of practical relevance, this could allow to have the client advisor in the same (virtual) room as the client to discuss the financial situation more collaboratively.
	
	\item \textbf{\gls{srq} 3:} In the current design, the actual financial transactions are only shown as part of a table. Additional exploratory analysis possibilities could be created by looking into the concept of the 'data forest' again. With a 3D visualisation of the individual transactions, new insights into the data might be found.
\end{itemize}
In addition, some other potential areas for future research that are not directly linked to any \glspl{srq} are the following:
\begin{itemize}[]
	\item With the current design only a single year/month/day can be active at any given time, whereas for the categories any combination is possible. With the addition of this approach the the temporal selection possibilities, some of the weaknesses as described in the findings could be further mitigated.
	
	\item With regards to the evaluation of the prototype, only one e-banking solution could be used for the scenario comparison as no other banking product allowed for such a categorisation of payments and also offered a working demo application. This might change in the future and would allow for a broader evaluation.
\end{itemize}
