%----------------------------------------------------------------------------------------
%	CHAPTER - CONCLUSION
%----------------------------------------------------------------------------------------

\chapter{Conclusion}

\label{ChapterConclusion}

The final chapter in this thesis covers the final phase of the \gls{dsr} cycle from \cite{Vaishnavi2008}, the \textit{Conclusion}. They classify this phase as the one where the results of the effort is consolidated and the knowledge gained categorized and documented \citep{Vaishnavi2008}.


%----------------------------------------------------------------------------------------
%	SECTION 1
%----------------------------------------------------------------------------------------

\section{Introduction}

In the first part of the conclusion, a summary of the research done in this thesis is provided. Then the main findings of the literature review and the design, implementation and evaluation of the prototype application are provided, as well as an answer to the \gls{mrq} from the first chapter. Finally, a suggestion for future research is given.


%----------------------------------------------------------------------------------------
%	SECTION 2
%----------------------------------------------------------------------------------------

\section{Summary}



The purpose of a conclusion is to answer your research question. Begin, with repeating your research question. However, don’t simply reiterate the research question, but integrate an explanation of it into the rest of the section’s discussion.

Then give the conclusions that you draw based on the results of your research (use the key results that are most relevant for answering your research question).

Finally, answer the main question and explain how you have come to this conclusion. Don’t just list the question with the answer below it, but carefully explain it and incorporate it into the rest of the text. Provide the raw observations and don’t interpret.

-----------------------------


The conclusion typically covers the following:

What was learned (this usually comes first)
What remains to be learned (directions for future research)
The shortcomings of what was done (evaluation)
The benefits, advantages, applications, etc. of the research (evaluation)
Recommendations


This basic pattern might help in terms of structure

Start para/mini-section with a summary of what was DONE ("I conducted an analysis of ABC)
Move into what was FOUND in terms of your analysis ("From this analysis, XYZ became apparent" -- this is also a 'summary')
Draw out implication 1, implication 2, etc. ("because of XYZ, we can now say PQR which leads to KLM")

THEN

Point to any gaps/problems ("However, we CANNOT say DEF")
And then to what needs to be done next ("So therefore we should look at GHI")
I have run out of alphabet.

%----------------------------------------------------------------------------------------
%	SECTION 3
%----------------------------------------------------------------------------------------

\section{Findings}

tbd
TODO: Maybe reflect back to literature and explain why author XYZ was right or wrong with his/her ideas.
TODO: Answer to Research Questions? Create a new one for the main part?


%----------------------------------------------------------------------------------------
%	SECTION 4
%----------------------------------------------------------------------------------------

\section{Future Research}

tbd


Include Support Tasks from VISM \newline
data forest for individual transactions \newline
select multiple months/days for the list of transactions \newline
include sub-categories \newline
speech recognition (e.g. Show me all transactions in the last seven days over CHF 100.) \newline
evaluate with other banking products  \newline