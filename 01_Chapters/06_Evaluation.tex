%----------------------------------------------------------------------------------------
%	CHAPTER - EVALUATION
%----------------------------------------------------------------------------------------

\chapter{Evaluation}

\label{ChapterEvaluation}

%----------------------------------------------------------------------------------------
%	SECTION 1
%----------------------------------------------------------------------------------------

\section{Introduction}

In the \textit{Evaluation} phase, the prototype is evaluated according to criteria from the \textit{Awareness of Problem} phase which either confirms or contradicts the initially defined hypothesis \citep{Vaishnavi2008}. The goal is to have an answer for the \gls{mrq}:
\begin{framed}
	\textit{\mrqtext}
\end{framed}
The evaluation of this prototype is based on the evaluation methods \textit{Testing} and \textit{Descriptive} as proposed by \cite{Hevner2004} and illustrated in Figure \ref{tbl:designevaluationmethods} in Chapter \ref{EvaluationMethodology}. These can be further broken down into \textit{Black Box} and \textit{White Box Testing}, as well as \textit{Informed Arguments} and \textit{Scenarios}. The following chapters are focussing on the individual methods.


%----------------------------------------------------------------------------------------
%	SECTION 2
%----------------------------------------------------------------------------------------

\section{Testing}

tbd


%-----------------------------------
%	SUBSECTION 1
%-----------------------------------
\subsection{White Box Testing}

tbd


%-----------------------------------
%	SUBSECTION 2
%-----------------------------------

\subsection{Black Box Testing}

tbd


%----------------------------------------------------------------------------------------
%	SECTION 3
%----------------------------------------------------------------------------------------

\section{Descriptive}

tbd
what is missing is guidance for how to perform the evaluation
TODO: Also check \cite{Peffers2012} for their definition of evaluation of a prototype


%-----------------------------------
%	SUBSECTION 1
%-----------------------------------
\subsection{Informed Arguments}

tbd


%-----------------------------------
%	SUBSECTION 2
%-----------------------------------

\subsection{Scenarios}

tbd




%----------------------------------------------------------------------------------------
%	SECTION 3
%----------------------------------------------------------------------------------------

\section{Conclusion}

tbd





