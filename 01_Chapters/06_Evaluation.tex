%----------------------------------------------------------------------------------------
%	CHAPTER - EVALUATION
%----------------------------------------------------------------------------------------

\chapter{Evaluation}

\label{ChapterEvaluation}

In the \textit{Evaluation} phase, the prototype is evaluated according to criteria from the \textit{Awareness of Problem} phase which either confirms or contradicts the initially defined hypothesis \citep{Vaishnavi2008}. The goal is to have an answer for the \gls{mrq}:
\begin{framed}
	\textit{\mrqtext}
\end{framed}

%----------------------------------------------------------------------------------------
%	SECTION 1
%----------------------------------------------------------------------------------------

\section{Introduction}

The evaluation of this prototype is based on the evaluation methods \textit{Testing} and \textit{Descriptive} as proposed by \cite{Hevner2004} and illustrated in Table \ref{tbl:designevaluationmethods} in Chapter \ref{EvaluationMethodology}. The \textit{Testing} evaluation method that focus on the functional aspect in executing the artefact interfaces to find defects and failures as well as the structural aspect for metric evaluation has been part of the artefact development and therefore is already covered in Chapter \ref{ChapterDevelopment}. The \textit{Descriptive} evaluation methods are broken further down into \textit{Scenarios} and \textit{Informed Arguments} which are discussed in the following chapters.


%----------------------------------------------------------------------------------------
%	SECTION 2
%----------------------------------------------------------------------------------------

\section{Descriptive: Scenarios}

\citet[p.4]{Peffers2012} define the evaluation method of a prototype as: \blockquote{Implementation of an artifact aimed at demonstrating the utility or suitability of the artifact.} They continue that a prototype can help to demonstrate the efficiency of a design, and to show that it works as intended, is useful for its intended purpose, or at least has the potential to be at an expected level of performance \citep{Peffers2012}. For this, a set of detailed scenarios are defined that demonstrate the utility of the prototype compared to the traditional way of executing the tasks:
\begin{itemize}[noitemsep,nolistsep]
	\item Checking for specific executed payment
	\item Comparing monthly household expenses with previous year
	\item Figuring out why the account balance is at zero in the middle of the month
	\item Exploring the financial impact of a changed living lifestyle
\end{itemize}
Excluded from all scenarios are the effort required in order to log in to e-banking or to get an export of the data set. The starting point is either the entry page of e-Banking or the empty, just loaded prototype application. In order to measure the efficiency changes with the prototype, the following metrics are considered:
\begin{itemize}[noitemsep,nolistsep]
	\item Number of (interacton) steps to get to desired information
	\item Required time
\end{itemize}


metrics: time, interaction-steps

exclude login / data download







Bank account is at zero in the middle of the month - figuring out why

Changed living lifestyle, what is the financial impact?





%-----------------------------------
%	SUBSECTION 1
%-----------------------------------
\subsection{tbd}

tbd


%-----------------------------------
%	SUBSECTION 2
%-----------------------------------

\subsection{tbd}

tbd


%----------------------------------------------------------------------------------------
%	SECTION 2
%----------------------------------------------------------------------------------------

\section{Descriptive: Informed Arguments}

tbd
what is missing is guidance for how to perform the evaluation
TODO: Also check \cite{Peffers2012} for their definition of evaluation of a prototype


%-----------------------------------
%	SUBSECTION 1
%-----------------------------------
\subsection{tbd}

tbd


%-----------------------------------
%	SUBSECTION 2
%-----------------------------------

\subsection{tbd}

tbd




%----------------------------------------------------------------------------------------
%	SECTION 3
%----------------------------------------------------------------------------------------

\section{Conclusion}

tbd





