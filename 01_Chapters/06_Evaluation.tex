%----------------------------------------------------------------------------------------
%	CHAPTER - EVALUATION
%----------------------------------------------------------------------------------------

\chapter{Evaluation}

\label{ChapterEvaluation}

%----------------------------------------------------------------------------------------
%	SECTION 1
%----------------------------------------------------------------------------------------

\section{Introduction}

In the \textit{Evaluation} phase, the prototype is evaluated according to criteria from the \textit{Awareness of Problem} phase which either confirms or contradicts the initially defined hypothesis \citep{Vaishnavi2008}. The goal is to have an answer for the \gls{mrq}:
\begin{framed}
	\textit{\mrqtext}
\end{framed}
The evaluation of this prototype is based on the evaluation methods \textit{Testing} and \textit{Descriptive} as proposed by \cite{Hevner2004} and illustrated in Figure \ref{tbl:designevaluationmethods} in Chapter \ref{EvaluationMethodology}. The \textit{Testing} evaluation method that focus on the functional aspect in executing the artefact interfaces to find defects and failures as well as the structural aspect for metric evaluation has been part of the artefact development and therefore is already covered in Chapter \ref{ChapterDevelopment}. The \textit{Descriptive} evaluation methods are broken further down into \textit{Informed Arguments} and \textit{Scenarios} which are discussed in the following chapters.


%----------------------------------------------------------------------------------------
%	SECTION 2
%----------------------------------------------------------------------------------------

\section{Testing}

In the first part, \textit{Black Box Testing}, the prototype is evaludated


				\cellcolor{green!25}\textbf{Functional (Black Box) Testing:} Execute artefact interfaces to discover failures and identify defects \\
				\hhline{|~|-|}
				& \cellcolor{green!25}\textbf{Structural (White Box) Testing:} Perform coverage testing of some metric (e.g. execution paths) in the artefact implementation \\

In the first part, Black Box Testing, the prototype will be evaluated under lab conditions where different performance and integration tests will be executed. Note that during the iterative prototyping phase units of the artefact has been tested regularly as part of the development to confirm correctness of the logical parts (White Box Testing).
tbd


%-----------------------------------
%	SUBSECTION 1
%-----------------------------------
\subsection{Black Box Testing}

tbd


%-----------------------------------
%	SUBSECTION 2
%-----------------------------------

\subsection{White Box Testing}

tbd


%----------------------------------------------------------------------------------------
%	SECTION 3
%----------------------------------------------------------------------------------------

\section{Descriptive}

tbd
what is missing is guidance for how to perform the evaluation
TODO: Also check \cite{Peffers2012} for their definition of evaluation of a prototype


%-----------------------------------
%	SUBSECTION 1
%-----------------------------------
\subsection{Informed Arguments}

tbd


%-----------------------------------
%	SUBSECTION 2
%-----------------------------------

\subsection{Scenarios}

tbd




%----------------------------------------------------------------------------------------
%	SECTION 3
%----------------------------------------------------------------------------------------

\section{Conclusion}

tbd





