%----------------------------------------------------------------------------------------
%	CHAPTER - EVALUATION
%----------------------------------------------------------------------------------------

\chapter{Evaluation} % Main chapter title

\label{ChapterEvaluation} % Change X to a consecutive number; for referencing this chapter elsewhere, use \ref{ChapterX}

%----------------------------------------------------------------------------------------
%	SECTION 1
%----------------------------------------------------------------------------------------

\section{Main Section 1}

Evaluation. Once constructed, the artifact is evaluated according to criteria that are always implicit and frequently made explicit in the Proposal (Awareness of Problem phase). Deviations from expectations, both quantitative and qualitative, are carefully noted and must be tentatively explained. That is, the evaluation phase contains an analytic sub-phase in which hypotheses are made about the behavior of the artifact. This phase exposes an epistemic fluidity that is in stark contrast to a strict interpretation of the positivist stance. At an equivalent point in positivist research, analysis either confirms or contradicts a hypothesis. Essentially, save for some consideration of future work as may be indicated by experimental results, the research effort is finished. For the design science researcher, by contrast, things are just getting interesting. Rarely, in design science research, are initial hypotheses concerning behavior completely borne out. Instead, the evaluation phase results and additional information gained in the construction and running of the artifact are brought together and fed back to another round of Suggestion (cf. the circumscription arrows of Figures 2.3 and 2.5). The explanatory hypotheses, which are quite broad, are rarely discarded; rather, they are modified to be in accord with the new observations. This suggests a new design, frequently preceded by new library research in directions suggested by deviations from theoretical performance. (Design science researchers seem to share Allen Newell’s concept [from cognitive science] of theories as complex, robust nomological networks.) This concept has been observed by philosophers of science in many communities (Lakatos, 1978); and working from it, Newell suggests that theories are not like clay pigeons, to be blasted to bits with the Popperian shotgun of falsification. Rather, they should be treated like doctoral students. One corrects them when they err, and is hopeful they can amend their flawed behavior and go on to be evermore useful and productive (Newell, 1990).
\cite{Vaishnavi2008}

Lorem ipsum dolor sit amet, consectetur adipiscing elit. Aliquam ultricies lacinia euismod. Nam tempus risus in dolor rhoncus in interdum enim tincidunt. Donec vel nunc neque. In condimentum ullamcorper quam non consequat. Fusce sagittis tempor feugiat. Fusce magna erat, molestie eu convallis ut, tempus sed arcu. Quisque molestie, ante a tincidunt ullamcorper, sapien enim dignissim lacus, in semper nibh erat lobortis purus. Integer dapibus ligula ac risus convallis pellentesque.

%-----------------------------------
%	SUBSECTION 1
%-----------------------------------
\subsection{Subsection 1}

Nunc posuere quam at lectus tristique eu ultrices augue venenatis. Vestibulum ante ipsum primis in faucibus orci luctus et ultrices posuere cubilia Curae; Aliquam erat volutpat. Vivamus sodales tortor eget quam adipiscing in vulputate ante ullamcorper. Sed eros ante, lacinia et sollicitudin et, aliquam sit amet augue. In hac habitasse platea dictumst.

%-----------------------------------
%	SUBSECTION 2
%-----------------------------------

\subsection{Subsection 2}
Morbi rutrum odio eget arcu adipiscing sodales. Aenean et purus a est pulvinar pellentesque. Cras in elit neque, quis varius elit. Phasellus fringilla, nibh eu tempus venenatis, dolor elit posuere quam, quis adipiscing urna leo nec orci. Sed nec nulla auctor odio aliquet consequat. Ut nec nulla in ante ullamcorper aliquam at sed dolor. Phasellus fermentum magna in augue gravida cursus. Cras sed pretium lorem. Pellentesque eget ornare odio. Proin accumsan, massa viverra cursus pharetra, ipsum nisi lobortis velit, a malesuada dolor lorem eu neque.

%----------------------------------------------------------------------------------------
%	SECTION 2
%----------------------------------------------------------------------------------------

\section{Main Section 2}

Sed ullamcorper quam eu nisl interdum at interdum enim egestas. Aliquam placerat justo sed lectus lobortis ut porta nisl porttitor. Vestibulum mi dolor, lacinia molestie gravida at, tempus vitae ligula. Donec eget quam sapien, in viverra eros. Donec pellentesque justo a massa fringilla non vestibulum metus vestibulum. Vestibulum in orci quis felis tempor lacinia. Vivamus ornare ultrices facilisis. Ut hendrerit volutpat vulputate. Morbi condimentum venenatis augue, id porta ipsum vulputate in. Curabitur luctus tempus justo. Vestibulum risus lectus, adipiscing nec condimentum quis, condimentum nec nisl. Aliquam dictum sagittis velit sed iaculis. Morbi tristique augue sit amet nulla pulvinar id facilisis ligula mollis. Nam elit libero, tincidunt ut aliquam at, molestie in quam. Aenean rhoncus vehicula hendrerit.