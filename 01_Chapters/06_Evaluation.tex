%----------------------------------------------------------------------------------------
%	CHAPTER - EVALUATION
%----------------------------------------------------------------------------------------

\chapter{Evaluation}

\label{ChapterEvaluation}

In the \textit{Evaluation} phase, the prototype is evaluated according to criteria from the \textit{Awareness of Problem} phase which either confirms or contradicts the initially defined hypothesis \citep{Vaishnavi2008}. The goal is to have an answer for the \gls{mrq}:
\begin{framed}
	\textit{\mrqtext}
\end{framed}

%----------------------------------------------------------------------------------------
%	SECTION 1
%----------------------------------------------------------------------------------------

\section{Introduction}

The evaluation of this prototype is based on the evaluation methods \textit{Testing} and \textit{Descriptive} as proposed by \cite{Hevner2004} and illustrated in Table \ref{tbl:designevaluationmethods} in Chapter \ref{EvaluationMethodology}. The \textit{Testing} evaluation method that focus on the functional aspect in executing the artefact interfaces to find defects and failures as well as the structural aspect for metric evaluation has been part of the artefact development and therefore is already covered in Chapter \ref{ChapterDevelopment}. The \textit{Descriptive} evaluation methods are broken further down into \textit{Scenarios} and \textit{Informed Arguments} which are discussed in the following chapters.


%----------------------------------------------------------------------------------------
%	SECTION 2
%----------------------------------------------------------------------------------------

\section{Descriptive: Scenarios}

\citet[p.4]{Peffers2012} define the evaluation method of a prototype as: \blockquote{Implementation of an artifact aimed at demonstrating the utility or suitability of the artifact}.



tbd
what is missing is guidance for how to perform the evaluation
TODO: Also check \cite{Peffers2012} for their definition of evaluation of a prototype



In this paper, we locate and examine DS research papers from IS and from engi-
neering research journals. We classify the papers to create taxonomies of the DS re- search artifact types and of the evaluation method types in use. Using these classifica- tions, we observe associations between types of artifacts and the methods used to evaluate them, showing which evaluation methods have been used to evaluate which types of artifacts.


Evalution Method
Prototype = Implementation of an artifact aimed at demonstrating the utility or suitability of the artifact

\cite{Peffers2012} 
The use of a prototype instantiation to demonstrate the efficacy of a design can
provide strong evidence when used to show that a design works as intended, is useful for its intended purpose, or has the potential to achieve an expected performance lev- el. The use of a specific instantiation suggests that the artifact should be evaluated on its directly observable performance, e.g., processing time.




%-----------------------------------
%	SUBSECTION 1
%-----------------------------------
\subsection{tbd}

tbd


%-----------------------------------
%	SUBSECTION 2
%-----------------------------------

\subsection{tbd}

tbd


%----------------------------------------------------------------------------------------
%	SECTION 2
%----------------------------------------------------------------------------------------

\section{Descriptive: Informed Arguments}

tbd
what is missing is guidance for how to perform the evaluation
TODO: Also check \cite{Peffers2012} for their definition of evaluation of a prototype


%-----------------------------------
%	SUBSECTION 1
%-----------------------------------
\subsection{tbd}

tbd


%-----------------------------------
%	SUBSECTION 2
%-----------------------------------

\subsection{tbd}

tbd




%----------------------------------------------------------------------------------------
%	SECTION 3
%----------------------------------------------------------------------------------------

\section{Conclusion}

tbd





