%----------------------------------------------------------------------------------------
%	CHAPTER - EVALUATION
%----------------------------------------------------------------------------------------

\chapter{Evaluation}

\label{ChapterEvaluation}

In the \textit{Evaluation} phase, the prototype is evaluated according to criteria from the \textit{Awareness of Problem} phase which either confirms or contradicts the initially defined hypothesis \citep{Vaishnavi2008}. The goal is to have an answer for the \gls{mrq}:
\begin{framed}
	\textit{\mrqtext}
\end{framed}

%----------------------------------------------------------------------------------------
%	SECTION 1
%----------------------------------------------------------------------------------------

\section{Introduction}

The evaluation of this prototype is based on the evaluation methods \textit{Testing} and \textit{Descriptive} as proposed by \cite{Hevner2004} and illustrated in Table \ref{tbl:designevaluationmethods} in Chapter \ref{EvaluationMethodology}. The \textit{Testing} evaluation method that focus on the functional aspect in executing the artefact interfaces to find defects and failures as well as the structural aspect for metric evaluation has been part of the artefact development and therefore is already covered in Chapter \ref{ChapterDevelopment}. The \textit{Descriptive} evaluation methods are broken further down into \textit{Scenarios} and \textit{Informed Arguments} which are discussed in the following chapters.


%----------------------------------------------------------------------------------------
%	SECTION 2
%----------------------------------------------------------------------------------------

\section{Descriptive: Scenarios}

% Definition of the Scenarios
\newcommand{\scenone}{Checking for specific financial transaction}
\newcommand{\scentwo}{Comparing monthly expenses with previous yearsn}
\newcommand{\scenthree}{Figuring out why  account balance is zero in the middle of the month}
\newcommand{\scenfour}{Tracking the monthly household expenses to not exceed the planned budget}

\citet[p.4]{Peffers2012} define the evaluation method of a prototype as: \blockquote{Implementation of an artifact aimed at demonstrating the utility or suitability of the artifact.} They continue that a prototype can help to demonstrate the efficiency of a design, and to show that it works as intended, is useful for its intended purpose, or at least has the potential to be at an expected level of performance \citep{Peffers2012}. For this, a set of detailed scenarios are defined that can demonstrate the utility of the prototype compared to the traditional way of executing the same tasks:
\begin{enumerate}[noitemsep,nolistsep]
	\item \scenone
	\item \scentwo
	\item \scenthree
	\item \scenfour
\end{enumerate}
Excluded from all scenarios are the steps required in order to log in to e-banking or to get an export of the data set for the prototype application. The starting point is either the entry page of e-Banking or the just started prototype application. In order to measure the efficiency changes with the prototype, the following metrics are considered:
\begin{itemize}[noitemsep,nolistsep]
	\item Number of steps to get to desired information
	\item Exclusivity of presented data
	\item Comprehensibility
\end{itemize}
While the first metric is of quantitative nature, the other two are qualitative. The exclusivity of the presented data is important in terms of whether only the required information is shown to the user and thus easy to identify, or if much more information is presented and the actually looked for part has to be searched for first. The comprehensibility focuses on how the data is presented and the difficulty to derive an exact answer to the question from it. \newline
At the current time, only the UBS e-Banking offers an integrated way for categorizing financial transaction on such a detailed level. While other banking solutions also allow to check for specific executed payments (Scenario 1), they lack the capabilities to provide a detailed enough answers to the other scenarios. Due to this, the evaluation is conducted just with the UBS e-Banking, but could be reproduced in future research with other banking solutions as well.


%-----------------------------------
%	SUBSECTION 1
%-----------------------------------
\subsection{Scenario 1}

\textbf{Title:} \scenone

\textbf{Situation:} On Friday before the weekend, a bank transfer was create to pay an outstanding invoice. The user now would like to see if this payment has already been executed and the money transferred away from his account.

%-----------------------------------
%	SUBSUBSECTION 1
%-----------------------------------

\subsubsection{Protoype Application}

tbd


%-----------------------------------
%	SUBSUBSECTION 2
%-----------------------------------

\subsubsection{UBS e-Banking}

tbd

%-----------------------------------
%	SUBSUBSECTION 1
%-----------------------------------

\subsubsection{Conclusion}

tbd


%-----------------------------------
%	SUBSECTION 2
%-----------------------------------

\subsection{Scenario 2}

\textbf{Title:} \scentwo

\textbf{Situation:} Since their first child was born, a family is wondering how their household and travelling expenses have changed compared to previous years.

%-----------------------------------
%	SUBSUBSECTION 1
%-----------------------------------

\subsubsection{Protoype Application}

tbd


%-----------------------------------
%	SUBSUBSECTION 2
%-----------------------------------

\subsubsection{UBS e-Banking}

tbd

%-----------------------------------
%	SUBSUBSECTION 1
%-----------------------------------

\subsubsection{Conclusion}

tbd




%-----------------------------------
%	SUBSECTION 3
%-----------------------------------

\subsection{Scenario 3}

\textbf{Title:} \scenthree

\textbf{Situation:}

%-----------------------------------
%	SUBSUBSECTION 1
%-----------------------------------

\subsubsection{Protoype Application}

tbd


%-----------------------------------
%	SUBSUBSECTION 2
%-----------------------------------

\subsubsection{UBS e-Banking}

tbd

%-----------------------------------
%	SUBSUBSECTION 1
%-----------------------------------

\subsubsection{Conclusion}

tbd




%-----------------------------------
%	SUBSECTION 4
%-----------------------------------

\subsection{Scenario 4}

\textbf{Title:} \scenfour

\textbf{Situation:}

%-----------------------------------
%	SUBSUBSECTION 1
%-----------------------------------

\subsubsection{Protoype Application}

tbd


%-----------------------------------
%	SUBSUBSECTION 2
%-----------------------------------

\subsubsection{UBS e-Banking}

tbd

%-----------------------------------
%	SUBSUBSECTION 1
%-----------------------------------

\subsubsection{Conclusion}

tbd




%----------------------------------------------------------------------------------------
%	SECTION 2
%----------------------------------------------------------------------------------------

\section{Descriptive: Informed Arguments}

tbd
what is missing is guidance for how to perform the evaluation
TODO: Also check \cite{Peffers2012} for their definition of evaluation of a prototype


%-----------------------------------
%	SUBSECTION 1
%-----------------------------------
\subsection{tbd}

tbd


%-----------------------------------
%	SUBSECTION 2
%-----------------------------------

\subsection{tbd}

tbd




%----------------------------------------------------------------------------------------
%	SECTION 3
%----------------------------------------------------------------------------------------

\section{Conclusion}

tbd





