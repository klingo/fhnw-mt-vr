%----------------------------------------------------------------------------------------
%	CHAPTER - EVALUATION
%----------------------------------------------------------------------------------------

\chapter{Evaluation}

\label{ChapterEvaluation}

%----------------------------------------------------------------------------------------
%	SECTION 1
%----------------------------------------------------------------------------------------

\section{Introduction}

Evaluation. Once constructed, the artifact is evaluated according to criteria that are always implicit and frequently made explicit in the Proposal (Awareness of Problem phase). Deviations from expectations, both quantitative and qualitative, are carefully noted and must be tentatively explained. That is, the evaluation phase contains an analytic sub-phase in which hypotheses are made about the behavior of the artifact. This phase exposes an epistemic fluidity that is in stark contrast to a strict interpretation of the positivist stance. At an equivalent point in positivist research, analysis either confirms or contradicts a hypothesis. Essentially, save for some consideration of future work as may be indicated by experimental results, the research effort is finished. For the design science researcher, by contrast, things are just getting interesting. Rarely, in design science research, are initial hypotheses concerning behavior completely borne out. Instead, the evaluation phase results and additional information gained in the construction and running of the artifact are brought together and fed back to another round of Suggestion (cf. the circumscription arrows of Figures 2.3 and 2.5). The explanatory hypotheses, which are quite broad, are rarely discarded; rather, they are modified to be in accord with the new observations. This suggests a new design, frequently preceded by new library research in directions suggested by deviations from theoretical performance. (Design science researchers seem to share Allen Newell’s concept [from cognitive science] of theories as complex, robust nomological networks.) This concept has been observed by philosophers of science in many communities (Lakatos, 1978); and working from it, Newell suggests that theories are not like clay pigeons, to be blasted to bits with the Popperian shotgun of falsification. Rather, they should be treated like doctoral students. One corrects them when they err, and is hopeful they can amend their flawed behavior and go on to be evermore useful and productive (Newell, 1990).
\cite{Vaishnavi2008}


%-----------------------------------
%	SUBSECTION 1
%-----------------------------------
\subsection{Subsection 1}

tbd


%-----------------------------------
%	SUBSECTION 2
%-----------------------------------

\subsection{Subsection 2}

tbd


%----------------------------------------------------------------------------------------
%	SECTION 2
%----------------------------------------------------------------------------------------

\section{Main Section 2}

tbd