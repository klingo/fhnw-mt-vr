%----------------------------------------------------------------------------------------
%	CHAPTER - INTRODUCTUION
%----------------------------------------------------------------------------------------

\chapter{Introduction} % Main chapter title

\label{ChapterIntroduction} % Change X to a consecutive number; for referencing this chapter elsewhere, use \ref{ChapterX}

%----------------------------------------------------------------------------------------
%	SECTION 1
%----------------------------------------------------------------------------------------

\section{Introduction}

With the public release of the Oculus Rift in March 2016 \citep{Oculus2016} and the HTC Vive in April 2016 \citep{Htcvive2016}, virtual reality has arrived at the consumer level and is all over the news. Although we only see the first iteration of consumer products, \cite{Gartner2015} predicts that within the next five to ten years, virtual reality will achieve mainstream adoption and enter the so called "Plateau of Productivity". Gesture Control is even a bit ahead of \gls{vr} and its mainstream adoption is already expected in the next two to five years \citep{Gartner2015}.

Simultaneously, we see massive advancements in the functionalities provided by banks to their clients. Contact-less payments and mobile banking is on the rise, and in 2014 UBS AG even won the "Master of Swiss Web" award and two gold medals for their new e-banking platform \citep{UBSAG2014}. But when looking at the financial situation, nothing has changed for a long time in terms of how it is presented to the users: always the same old bar- and pie charts.

\begin{figure}[h]
	\begin{center}
		\includegraphics[width=7cm]{03_Figures/06_Introduction/UBSAG2016_SpendingAnalysis2.png}
		\includegraphics[width=7cm]{03_Figures/06_Introduction/UBSAG2016_SpendingAnalysis.png}
		\caption[Different visualisations of the spending analysis in UBS e-banking demo]{Different visualisations of the spending analysis in UBS e-banking demo \citep{UBSAG2016}}
		\label{fig:ubsspendinganalysis}
	\end{center}
\end{figure}

The focus of this master thesis is on the possibilities of enhanced user interaction with the data about our financials situation in \gls{vr} by utilizing not only hand gestures, but making use of the additional sensor information of gesture controllers and 360° motion tracking

In the following sub-chapters, more background information about the topic is given as well as the definition of the problem statement. Based on this, the thesis statement is proposed and research questions are derived from. Following these, the delineations and limitations are presented before this chapter is closed with the structure of the thesis and a brief overview of the chapters and their correlations

TODO: Start with relevance to the real world!
TODO: Intro is too short!


%----------------------------------------------------------------------------------------
%	SECTION 2
%----------------------------------------------------------------------------------------

\section{Background}

\gls{vr} has been around us for a few decades already, but only in the last couple of years with increasingly fast hardware and reduced costs, \gls{vr} made rapid advancements \citep{vrs2015}. Computers and even smartphones are now powerful enough to provide good \gls{vr} experiences.\newline
Very dominant at the moment are the entertainment purposes, for which \gls{vr} is currently mainly utilized, but it can also go beyond that and provide profound educational experiences if curated with the right content \citep{Safrudin2015}. \gls{vr} even should be seen as disruptive, as a \textbf{game changer} that can become accessible to everyone and will be very promising for high-tech enterprises that focus on retail and banking \citep{Safrudin2015}. \newline
Every year, Gartner publishes their hype cycle on different topics, including the \textit{Emerging Technology Hype Cycle} from 2015 (Figure \ref{fig:hypecycle}). This shows that if Gartner is to be believed, \gls{vr} has the phase of over-excitement and unrealistic expectations already behind it and is close to the point where it is widely understood by the general public. A bit further than \gls{vr} itself is Gesture Control that according to \cite{Gartner2015} will already reach mainstream adoption in the next two to five years.
\begin{figure}[h]
	\begin{center}
		\includegraphics[width=14cm]{03_Figures/03_Gartner/Gartner_EmergingTech2015.png}
		\caption[Emerging Technology Hype Cycle]{Emerging Technology Hype Cycle \citep{Gartner2015b}}
		\label{fig:hypecycle}
	\end{center}
\end{figure} \newline
There are different interaction methods (i.e. devices to interact with the \gls{ve}) that each provide advantages and disadvantages in the execution of tasks belonging to one of three groups of interaction patterns: \textit{Travel}, \textit{Selection} and \textit{Manipulation} \citep{Bowman2002}. 


%----------------------------------------------------------------------------------------
%	SECTION 3
%----------------------------------------------------------------------------------------

\section{Problem Statement}

\gls{vr} becomes more important in our lives in the future and it is crucial to have effective and efficient means to interact with the \gls{ve} in order to secure the success of this technology. Research has been done on how to travel within the \gls{ve} and how to select objects, but not much has been done yet on the side of manipulation. While there are some guidelines and recommendations, they are all of theoretical nature and no practical recommendations for specific technologies are available.


%----------------------------------------------------------------------------------------
%	SECTION 4
%----------------------------------------------------------------------------------------

\section{Thesis Statement}

The thesis statement is as follows:
\begin{framed}
	\textit{The interaction with multidimensional data in virtual reality can be enhanced by utilizing gesture controllers and 360° motion tracking.}
\end{framed} \label{TS}


%----------------------------------------------------------------------------------------
%	SECTION 5
%----------------------------------------------------------------------------------------

\section{Research question}

To address the problem statement, research questions are used to break down the research into smaller parts that can be examined individually and allow a view at the nature of the problem from different perspectives.
In this chapter, the \gls{mrq} as well as the \glspl{srq} are formulated. \newline
The main research question, derived from the thesis statement, is:
\begin{framed}
	\textit{How can the interaction with multidimensional data in virtual reality be enhanced by utilizing gesture controllers and 360° motion tracking?}
\end{framed} \label{MRQ}
From the MRQ, the SRQs can be derived and are defined as follows:
\begin{framed}
	\textit{SRQ 1: Which methods of user input in virtual reality are researched and what are their advantages and disadvantages?}
\end{framed} \label{SRQ1}
\begin{framed}
	\textit{SRQ 2: Which ways of interaction for multi-dimensional data exist and what are their strengths and weaknesses?}
\end{framed} \label{SRQ2}
\begin{framed}
	\textit{SRQ 3: Which traditional strategies for visualization and manipulation of 2D data have been applied and enhanced for 3D space in virtual reality?}
\end{framed} \label{SRQ3}
 

%----------------------------------------------------------------------------------------
%	SECTION 6
%----------------------------------------------------------------------------------------

\section{Research Objective}

TODO: Rewrite this part, as it contains the solution?
The research objective of this master thesis is to enhance the interaction with data in virtual reality by using sensor information from gesture controllers and 360° motion tracking.\newline
This objective will be addressed by first conducting a literature review on the research that has already been done in this field. Based on this a prototype will be designed and implemented that will make use of the to be proposed interaction patterns. This also acts as a verification that the interaction patterns indeed are feasible for the currently available technology.
TOPDDO: re-check mentioned interaction pattersn (ch2 erst)
%----------------------------------------------------------------------------------------
%	SECTION 7
%----------------------------------------------------------------------------------------

% \section{Short Overview}


%----------------------------------------------------------------------------------------
%	SECTION 8
%----------------------------------------------------------------------------------------

\section{Delineations and Limitations}

In this thesis, the focus lies on interaction possibilities with gesture controllers and 360° motion tracking. Other means of input such as from speech recognition is not part of the research.
In regards of the design and implementation of the prototype, the artefact will be implemented in Unity3D 5.4 with the SteamVR framework. The hardware in scope for the verification is the HTC Vive since it currently is the only consumer product with the  technical capabilities of gesture controllers and 360° motion tracking. Although focusing on the HTC Vive, due to the SteamVR framework functionality (Figure \ref{fig:steamvr}), the reuse of the research for other hardware should be possible in the future. \newline
Furthermore, with the design and development of a prototype, this thesis focuses more on the technical feasibility than the psychological aspects of usability, user experience, or productivity measurements. 
\begin{figure}[h]
	\begin{center}
		\includegraphics[width=14cm]{03_Figures/04_Valve/OpenVR_SteamVR.png}
		\caption[Steam VR Unity Plugin]{Steam VR Unity Plugin (adopted from \cite{Valve2016})}
		\label{fig:steamvr}
	\end{center}
\end{figure}

TODO: check to re-add:
 The regular game controllers have been excluded in this research since they basically are not much different from mouse and keyboard as they do not provide any added interaction functionality.

%----------------------------------------------------------------------------------------
%	SECTION 9
%----------------------------------------------------------------------------------------

% \section{Underlying Assumptions}


%----------------------------------------------------------------------------------------
%	SECTION 10
%----------------------------------------------------------------------------------------

% \section{Definition of terms and concepts}


%----------------------------------------------------------------------------------------
%	SECTION 11
%----------------------------------------------------------------------------------------

% \section{Significance}


%-----------------------------------
%	SUBSECTION 1
%-----------------------------------

% \subsection{Theoretical}


%-----------------------------------
%	SUBSECTION 2
%-----------------------------------

% \subsection{Practical}


%----------------------------------------------------------------------------------------
%	SECTION 12
%----------------------------------------------------------------------------------------

\section{Thesis structure and brief chapter overviews}

The thesis map in Figure \ref{fig:thesismap} shows the structure of this thesis and the relations of the individual chapters. \newline
In chapter \ref{ChapterIntroduction}, an introduction on the topic and the research problem is given. From this the thesis statement, research questions and the research objectives are derived. Following in chapter \ref{ChapterLiteratureReview} are the literature review of different methods for user input and interaction patterns in \gls{vr} as well as the enhancement of those. The research methodology follows in chapter \ref{Research Method} where the philosophy, approach and strategy are described. \newline
\begin{figure}[pt]
	\begin{center}
		\includegraphics[width=14cm]{03_Figures/06_Introduction/ThesisMap.png}
		\caption{Thesis Map}
		\label{fig:thesismap}
	\end{center}
\end{figure}

TODO: UPDATE THIS GRAPHIC!!!

%----------------------------------------------------------------------------------------
%	SECTION 13
%----------------------------------------------------------------------------------------

% \section{Any other institutional requirement not covered here}



