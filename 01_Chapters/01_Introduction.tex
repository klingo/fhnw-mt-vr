%----------------------------------------------------------------------------------------
%	CHAPTER - INTRODUCTUION
%----------------------------------------------------------------------------------------

\chapter{Introduction} % Main chapter title

\label{ChapterIntroduction} % Change X to a consecutive number; for referencing this chapter elsewhere, use \ref{ChapterX}

%----------------------------------------------------------------------------------------
%	SECTION 1
%----------------------------------------------------------------------------------------

\section{Introduction}

This master thesis provides on overview of the current state of research regarding the usage of gesture controllers and 360° motion tracking to enhance data interaction in \gls{vr}. The specific data-set that in focus of this thesis, are categorized financial expenses from individuals.

With the public release of the Oculus Rift in March 2016 \citep{Oculus2016} and the HTC Vive in April 2016 \citep{Htcvive2016}, \gls{vr} has arrived at the consumer level and is all over the news. Although we only see the first iteration of consumer products, \cite{Gartner2015} predicts that within the next five to ten years, virtual reality will achieve mainstream adoption and enter the so called "Plateau of Productivity". Gesture Control is even a bit ahead of \gls{vr} and its mainstream adoption is already expected in the next two to five years \citep{Gartner2015}. Together with the new \gls{hmd}, also the ways to interact with the virtual world have been massively improved, such as very accurate gesture controllers or full 360° motion tracking, which all are just becoming part of research now. Since these big advancements in hardware happened only very recently, the software part first has to catch up with all the new possibilities that became available. It became very easy to create 3D visualisations, but the big questions remains about why we should do that. \cite{Ware2012} sees clear clear advantages in the conventional two-dimensional techniques with bar charts and scatter plots, since the powerful pattern-finding mechanism of our brain also works in 2D. Furthermore, there are well established and effective (2D) data representations that are also very easy to be included in any kind of two-dimensional medium such as a book or a report \citep{Ware2012}. On the other side, \cite{Ware2012} also points out that we live in a three-dimensional word and interact with it so frequently that our brains got very much used to it. It is also important to understand that ultimately a 2D space is part of 3D space and thus any 2D object can always be flattened out and presented as such in 3D space. This however does not mean, that this always is a good idea. \cite{Kwon2015} stated that in a 2D environment/space (like a regular screen) it makes sense to use 2D objects for the presentation, whereas in a \gls{vr} environment where the viewing direction can be freely chosen by the user, the whole three dimensional space should be utilized. It generally can be said that we are only at the beginning of understanding how to effectively use these new means. Chapter \ref{SectionLiteratureReviewSRQ1} and \ref{SectionLiteratureReviewSRQ2} in the literature review will further discuss these new possibilities.

Simultaneously, we see massive advancements in the functionalities provided by banks to their clients. Contact-less payments and mobile banking are on the rise, and in 2014 UBS AG even won the "Master of Swiss Web" award and two gold medals for their new e-banking platform \citep{UBSAG2014}. But when client want to look at the financial situation, nothing has changed for a long time in terms of how it is presented to the users: always the same old bar- and pie charts. Figure \ref{fig:ubsspendinganalysis} gives an impression of how exemplary data is visualized on a 2D screen. Many challenges and requirements have to be covered nowadays. Clients want to immediately see and understand their financial situation, they want to be able to make comparisons and see trends. This explorative analysis is currently only possible with limitations, also due to the restriction of only two (real) axis that are possible on a display. \cite{Jamieson2007} see that it becomes increasingly difficult to get a meaningful understanding of the big amount of data that is still presented with traditional methods. They continue that these methods will sooner or later come to their limits and new visualisation methods have to be established. Chapter \ref{SectionLiteratureReviewSRQ3} in the literature review addresses the currently researched methods for data visualization in more detail.
\begin{figure}[h]
	\begin{center}
		\includegraphics[width=7cm]{03_Figures/06_Introduction/UBSAG2016_SpendingAnalysis2.png}
		\includegraphics[width=7cm]{03_Figures/06_Introduction/UBSAG2016_SpendingAnalysis.png}
		\caption[Different visualisations of the spending analysis in UBS e-banking demo]{Different visualisations of the spending analysis in UBS e-banking demo \citep{UBSAG2016}}
		\label{fig:ubsspendinganalysis}
	\end{center}
\end{figure}

The focus of this master thesis is on the new possibilities of visualising and interacting with data about our financial situation in \gls{vr} by utilizing the latest available (consumer) \gls{vr}-hardware and research about data visualisation. \newline
In the following sub-chapters, more background information about the topic is given as well as the definition of the problem statement. Based on this, the thesis statement is proposed and research questions are derived from. Following these, the delineations and limitations are presented before this chapter is closed with the structure of the thesis and a brief overview of the chapters and their correlations.


%----------------------------------------------------------------------------------------
%	SECTION 2
%----------------------------------------------------------------------------------------

\section{Background}

\gls{vr} has been around us for a few decades already, but only in the last couple of years with increasingly fast hardware and reduced costs, \gls{vr} made rapid advancements \citep{vrs2015}. Computers and even smartphones are now powerful enough to provide good \gls{vr} experiences.\newline
Very dominant at the moment are the entertainment purposes, for which \gls{vr} is currently mainly utilized, but it can also go beyond that and provide profound educational experiences if curated with the right content \citep{Safrudin2015}. \gls{vr} even should be seen as disruptive, as a \textbf{game changer} that can become accessible to everyone and will be very promising for high-tech enterprises that focus on retail and banking \citep{Safrudin2015}. \newline
Every year, Gartner publishes their hype cycle on different topics, including the \textit{Emerging Technology Hype Cycle} from 2015 (Figure \ref{fig:hypecycle}). This shows that if Gartner is to be believed, \gls{vr} has the phase of over-excitement and unrealistic expectations already behind it and is close to the point where it is widely understood by the general public. A bit further than \gls{vr} itself is Gesture Control that according to \cite{Gartner2015} will already reach mainstream adoption in the next two to five years.
\begin{figure}[h]
	\begin{center}
		\includegraphics[width=14cm]{03_Figures/03_Gartner/Gartner_EmergingTech2015.png}
		\caption[Emerging Technology Hype Cycle]{Emerging Technology Hype Cycle \citep{Gartner2015b}}
		\label{fig:hypecycle}
	\end{center}
\end{figure} \newline
There are different interaction methods (i.e. devices to interact with the \gls{ve}) that each provide advantages and disadvantages in the execution of tasks belonging to one of three groups of interaction patterns: \textit{Travel}, \textit{Selection} and \textit{Manipulation} \citep{Bowman2002}. 


%----------------------------------------------------------------------------------------
%	SECTION 3
%----------------------------------------------------------------------------------------

\section{Problem Statement}

\gls{vr} becomes more important in our lives in the future and it is crucial to have effective and efficient means to interact with the \gls{ve} in order to secure the success of this technology. Research has been done on how to travel within the \gls{ve} and how to select objects, but not much has been done yet on the side of manipulation. While there are some guidelines and recommendations, they are all of theoretical nature and no practical recommendations for specific technologies are available. \newline
Even in 2016, the visualisation of data in a nutshell is still looking the same like a decade ago. There were only minor advancements that all are still focusing on the (limited) 2D space while only little research has been made in how to utilize the (real) third dimension in \gls{vr}.


%----------------------------------------------------------------------------------------
%	SECTION 4
%----------------------------------------------------------------------------------------

\section{Thesis Statement}

TODO: Choose one of the thesis statements

The thesis statement is as follows:
\begin{framed}
	\textit{The interaction with multidimensional data in virtual reality can be enhanced by utilizing gesture controllers and 360° motion tracking.}
\end{framed} \label{TS}

\begin{framed}
	\textit{The interaction with categorized financial data can be enhanced by utilizing gesture controllers and 360° motion tracking in virtual reality.}
\end{framed}

\begin{framed}
	\textit{The interaction with categorized financial data can be enhanced by utilizing new input methods and interaction patterns in virtual reality.}
\end{framed}

%----------------------------------------------------------------------------------------
%	SECTION 5
%----------------------------------------------------------------------------------------

\section{Research question}

TODO: Remove controllers and 360° info (part of solution?)
TODO: Choose one of the MRQs

To address the problem statement, research questions are used to break down the research into smaller parts that can be examined individually and allow a view at the nature of the problem from different perspectives. In this chapter, the \gls{mrq} as well as the \glspl{srq} are formulated. \newline
The main research question, derived from the thesis statement, is:
\begin{framed}
	\textit{How can the interaction with multidimensional data in virtual reality be enhanced by utilizing gesture controllers and 360° motion tracking?}
\end{framed} \label{MRQ}
\begin{framed}
	\textit{How can the interaction with categorized financial data be enhanced by utilizing gesture controllers and 360° motion tracking in virtual reality?}
\end{framed}
\begin{framed}
	\textit{How can the interaction with categorized financial data be enhanced by utilizing new input methods and interaction patterns in virtual reality?}
\end{framed}
From the \gls{mrq}, the \glspl{srq} can be derived and are defined as follows:
\begin{framed}
	\textit{\gls{srq} 1: Which methods of user input in virtual reality are researched and what are their advantages and disadvantages?}
\end{framed} \label{SRQ1}
\begin{framed}
	\textit{\gls{srq} 2: Which ways of interaction for multi-dimensional data exist and what are their strengths and weaknesses?}
\end{framed} \label{SRQ2}
\begin{framed}
	\textit{\gls{srq} 3: Which traditional strategies for visualization and manipulation of 2D data have been applied and enhanced for 3D space in virtual reality?}
\end{framed} \label{SRQ3}
 

%----------------------------------------------------------------------------------------
%	SECTION 6
%----------------------------------------------------------------------------------------

\section{Research Objective}

The research objective of this master thesis is to enhance the visualisation and interaction with categorized financial data in virtual reality by utilizing the latest (consumer) \gls{vr}-hardware in terms of display and input devices. \newline
This objective will be addressed by first conducting a literature review on the research that has already bee done in the fields of input methods and interaction patterns, as well as data visualisation. Based on this, a dedicated approach for the visualisation and the corresponding interaction patterns will be defined and applied in a prototype application. This also acts as a verification that the visualisation approach and interaction patterns indeed are feasible.


%----------------------------------------------------------------------------------------
%	SECTION 7
%----------------------------------------------------------------------------------------

% \section{Short Overview}


%----------------------------------------------------------------------------------------
%	SECTION 8
%----------------------------------------------------------------------------------------

\section{Delineations and Limitations}

In this thesis, the focus lies on publicly available consumer \gls{vr} hardware and no own display or input devices will be developed. In regards of the design and implementation of the prototype, the artefact will be implemented in Unity3D 5.5.0 with the SteamVR framework. The decision for this framework is based on the fact that it acts as a consolidation layer for many different \gls{vr} devices (see Figure \ref{fig:steamvr}) and thus should provide the highest chance of re-usability for current and future \gls{vr} hardware. \newline
Furthermore, with the design and development of a prototype, this thesis focuses more on the technical feasibility than the psychological aspects of usability, user experience, or productivity measurements. 
\begin{figure}[h]
	\begin{center}
		\includegraphics[width=14cm]{03_Figures/04_Valve/OpenVR_SteamVR.png}
		\caption[Steam VR Unity Plugin]{Steam VR Unity Plugin (adopted from \cite{Valve2016})}
		\label{fig:steamvr}
	\end{center}
\end{figure}


%----------------------------------------------------------------------------------------
%	SECTION 9
%----------------------------------------------------------------------------------------

% \section{Underlying Assumptions}


%----------------------------------------------------------------------------------------
%	SECTION 10
%----------------------------------------------------------------------------------------

% \section{Definition of terms and concepts}


%----------------------------------------------------------------------------------------
%	SECTION 11
%----------------------------------------------------------------------------------------

% \section{Significance}


%-----------------------------------
%	SUBSECTION 1
%-----------------------------------

% \subsection{Theoretical}


%-----------------------------------
%	SUBSECTION 2
%-----------------------------------

% \subsection{Practical}


%----------------------------------------------------------------------------------------
%	SECTION 12
%----------------------------------------------------------------------------------------

\section{Thesis structure and brief chapter overviews}

The thesis map in Figure \ref{fig:thesismap} shows the structure of this thesis and the relations of the individual chapters. \newline
In chapter \ref{ChapterIntroduction}, an introduction on the topic and the research problem is given. From this the thesis statement, research questions and the research objectives are derived. Following in chapter \ref{ChapterLiteratureReview} are the literature review of different methods for user input and interaction patterns in \gls{vr} as well as the enhancement of those. The research methodology follows in chapter \ref{Research Method} where the philosophy, approach and strategy are described. \newline
\begin{figure}[pt]
	\begin{center}
		\includegraphics[width=14cm]{03_Figures/06_Introduction/ThesisMap.png}
		\caption{Thesis Map}
		\label{fig:thesismap}
	\end{center}
\end{figure}

TODO: UPDATE THIS GRAPHIC!!!

%----------------------------------------------------------------------------------------
%	SECTION 13
%----------------------------------------------------------------------------------------

% \section{Any other institutional requirement not covered here}



