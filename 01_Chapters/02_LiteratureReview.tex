%----------------------------------------------------------------------------------------
%	CHAPTER - LITERATURE REVIEW
%----------------------------------------------------------------------------------------

\chapter{Literature Review} % Main chapter title

\label{ChapterLiteratureReview} % Change X to a consecutive number; for referencing this chapter elsewhere, use \ref{ChapterX}

In the literature review chapter, an overview of existing research found in secondary literature, that is relevant to the thesis statement of this paper, is given. Every \gls{srq} is discussed in its own subchapter where its relevance to the \gls{mrq} is elaborated. Each subchapter ends with a short conclusion and answers to the corresponding research question are given.
Figure XXX shows the correlation between the subchapters and the research questions. In chapter \ref{SectionLiteratureReviewSRQ1}, XXXX. The next chapter \ref{SectionLiteratureReviewSRQ2} XXXX. By XXXXX, the third SRQ is covered in chapter \ref{SectionLiteratureReviewSRQ3}. To wrap everything up, a conclusion of the literature review is presented in chapter \ref{SectionLiteratureReviewConclusion} which builds the base for the research design in chapter \ref{Research Method}.

% TODO: ADD FIGURE AND COMPLETE


%----------------------------------------------------------------------------------------
%	SECTION 1
%----------------------------------------------------------------------------------------

\section{Existing Interaction Patterns with Virtual Reality}

\label{SectionLiteratureReviewSRQ1}

%-----------------------------------
%	SUBSECTION 1
%-----------------------------------
\subsection{Introduction}

\gls{hci} itself has been worked and research on approximately from the 1950s onwards, where the main focus was on the direct manipulation of graphical objects, the mouse as well as gesture recognition \citep{Myers1998}. During this time however, gesture recognition was rather understood as devices that work with pen-based input devices and thus can recognize patterns that are drawn with these pens \citep{Myers1998}. In the regular interactiion between human and machines this is still fairly sufficient as even today we are still fully relying on having a mouse/trackpad/touchscreen and a keyboard to interact with our computers. \newline
Virtual reality changes this since quite a bit as waering a \gls{hmd} with their own displays obstructs the view on the so far used physical input devices. New solutions had to be found for this changed situation. An overview on the researched interaction patterns with virtual reality is shown in the following subchapter.


%-----------------------------------
%	SUBSECTION 2
%-----------------------------------

\subsection{Overview}

\subsubsection{Controllers}

\subsubsection{Hand Gestures}

We highlight results drawn from a study on pointing and draw conclusions for the implementation of pointing-based conversational interactions in partly immersive virtual reality.
\cite{Pfeiffer2008}

The environment constructed in this research allows a user to communicate by talking and showing gestures
to a personified agent in virtual environment. A user can use his/her finger to point at a virtual object and ask the agent to manipulate the virtual object.
\cite{Uchino2008}

The Hand gesture recognition system based interface proposed and implemented in this paper consists of a detection, tracking and recognition module.
Hand gesture communication based vocabulary offers many variations ranging from simple action of using our finger to point at to using hands for moving objects around to the rather complex one like expression of the feelings. The proposed hand gesture recognition system offers intensions to traditional input devices for interaction with the virtual environments
\cite{Rautaray2011}

Oculus Rift can track head movement and change view point follow it. Leap Motion is in - air controller that can track hand gesture of the user. The combination of them will make users feel like immerse to VR. Users can move avatar any way in VR by their hand interact through the system via these devices. We introduce a new interactive hand gesture system with palm normal for control steering develop by the game engine Unity3D applies synchronization of Oculus Rift and Leap Motion.
\cite{Khundam2015a}

It allows user to interact with the virtual reality system with the static and stroke hand gesture along with speech. This multimodal interaction technique is able to perform few functions such as select, move, scale, rotate, copy, mirror, delete, check the shape, check and change the colour of an object.
\cite{Chun2015}

There are some limitations of the multimodal interaction. For example, when the hand is not tracked properly, the stroke gesture recognizer cannot recognize the stroke gesture command. User also has to perform the stroke command in certain speed level.
\cite{Chun2015}

\subsubsection{Speech Recognition}

A system for the visualization of three-dimensional anatomical data, derived from magnetic resonance imaging (MRI) or computed tomography (CT), enables the physician to navigate through and interact with the patient's 3D scans in a virtual environment. This paper presents the multimodal human-machine interaction focusing the speech input. For the concerned task, a speech understanding front-end using a special kind of semantic decoder was successfully adopted
\cite{Muller1998}

The environment constructed in this research allows a user to communicate by talking and showing gestures
to a personified agent in virtual environment. A user can use his/her finger to point at a virtual object and ask the agent to manipulate the virtual object.
\cite{Uchino2008}

It allows user to interact with the virtual reality system with the static and stroke hand gesture along with speech. This multimodal interaction technique is able to perform few functions such as select, move, scale, rotate, copy, mirror, delete, check the shape, check and change the colour of an object.
\cite{Chun2015}


\subsubsection{Physical Placement of Interactive Objects}

The force feedback subsystem uses robotic positioning to place an assortment of knobs and switches into position to be touched; the user's hand trajectory is extrapolated and the correct type of control is placed just in time to be actuated.
\cite{Latham1997}


\subsubsection{Full Body Tracking}

We present a video1 of a VR demo called TurboTuscany, where we employ such controllers; our demo combines a Kinect controlled full body avatar with Oculus Rift head-mounted-display [2]. We implemented three positional head tracking schemes that use Kinect, Razer Hydra, and PlayStation (PS) Move controllers.
\cite{Takala2014}









%-----------------------------------
%	SUBSECTION 3
%-----------------------------------

\subsection{Visual Information Seeking Mantra}

blub



%-----------------------------------
%	SUBSECTION 4
%-----------------------------------

\subsection{Conclusion}

blub



%----------------------------------------------------------------------------------------
%	SECTION 2
%----------------------------------------------------------------------------------------

\section{SRQ 2}

\label{SectionLiteratureReviewSRQ2}

blub



%----------------------------------------------------------------------------------------
%	SECTION 3
%----------------------------------------------------------------------------------------

\section{SRQ 3}

\label{SectionLiteratureReviewSRQ3}

blub




%----------------------------------------------------------------------------------------
%	SECTION 4
%----------------------------------------------------------------------------------------

\section{Conclusion}

\label{SectionLiteratureReviewConclusion}

blub

