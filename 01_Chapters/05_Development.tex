%----------------------------------------------------------------------------------------
%	CHAPTER - DEVELOPMENT
%----------------------------------------------------------------------------------------

\chapter{Development}

\label{ChapterDevelopment}

%----------------------------------------------------------------------------------------
%	SECTION 1
%----------------------------------------------------------------------------------------

\section{Introduction}

Development. The Tentative Design is further developed and implemented in this phase. Elaboration of the Tentative Design into complete design requires creative effort. The techniques for implementation will of course vary, depending on the artifact to be constructed. An algorithm may require construction of a formal proof. An expert system embodying novel assumptions about human cognition in an area of interest will require software development, probably using a high-level package or tool. The implementation itself can be very pedestrian and need not involve novelty beyond the state-of-practice for the given artifact; the novelty is primarily in the design, not the construction of the artifact.
\cite{Vaishnavi2008}


TODO:
- explain the whole setup, how to get everything up and running
- 


%----------------------------------------------------------------------------------------
%	SECTION 2
%----------------------------------------------------------------------------------------

\section{Technical Setup}

TODO: Hardware selection still has to be covered!

HTC Vive (HMD; 2 Lighthouse, 2 Gesture Controllers)
Unity 3D 5.5
SteamVR Plugin


%----------------------------------------------------------------------------------------
%	SECTION 3
%----------------------------------------------------------------------------------------

\section{to be defined}


%% IDEA FOR PROTOTYPE

- Start with a small table on which a house etc is visualized. floating above is year (+month)
- Each object represents one category from the financial expenses.
- maybe size indicates the overall amount?
- The colours depends on the difference between my planned expenses (or average expenses) and the actual expenses. less = green, about the same = yellow-(green-)ish, slightly above = orange, above = red.
- By clicking on one of the objects, it gets highlighted and a line-chart appears showing the expenses
A) show individual transactions for the given month up until the threshold
B) show individual months until the threshold (+forecast?)
- show multiple lines for the different sub-categories (enable/disable) plus the total
- clicking on an entry of the line chart displays details about transaction (amount, location), or the month (amount) in an overlay
- switching between single-month and year view by... using the touchpad (up/down clicks)
- navigating through months/years by... a using the touchpad! (left/right clicks)
- resetting the view to start by... clicking on the select button

- maybe outside as rotating rings: the individual bank accounts?




%-----------------------------------
%	SUBSECTION 1
%-----------------------------------
\subsection{Subsection 1}

tbd


%-----------------------------------
%	SUBSECTION 2
%-----------------------------------

\subsection{Subsection 2}

tbd


%----------------------------------------------------------------------------------------
%	SECTION 2
%----------------------------------------------------------------------------------------

\section{Main Section 2}

tbd