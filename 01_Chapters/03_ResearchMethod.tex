%----------------------------------------------------------------------------------------
%	CHAPTER - RESEARCH METHOD
%----------------------------------------------------------------------------------------

\chapter{Research Method} % Main chapter title

\label{Research Method} % Change X to a consecutive number; for referencing this chapter elsewhere, use \ref{ChapterX}

%----------------------------------------------------------------------------------------
%	SECTION 1
%----------------------------------------------------------------------------------------

\section{Introduction}

In this chapter, the research methodology of this thesis is described. An interpretive research philosophy is assumed for this thesis. \cite{Hevner2010} defined different research strategies out of which design science is applied.


%----------------------------------------------------------------------------------------
%	SECTION 2
%----------------------------------------------------------------------------------------

\section{Philosophy}

The research philosophy shows how the researcher views the world and the knowledge that has to be developed as well as its nature.

\cite{Saunders2009} defines four different research philosophies:
\begin{itemize}[noitemsep,nolistsep]
	\item Positivism \textit{(only observable phenomena will lead
		to the production of credible data)}
	\item Realism \textit{(do objects exist independently of our
		knowledge of their existence?)}
	\item Interpretivism \textit{(understanding differences 
		between humans as social actors)}
	\item Pragmatism \textit{(do you have to adopt one position?)}
\end{itemize}

In addition, \cite{Saunders2009} also lists three major ways of how to think about the different research philosophies:
\begin{itemize}[noitemsep,nolistsep]
	\item Epistemology \textit{(the view of the nature of reality or being)}
	\item Ontology \textit{(the view of what constitutes acceptable knowledge)}
	\item Axiology \textit{(the view of the role of values in research)}
\end{itemize}



to be added



An interpretive research philosophy as defined by \cite{Vaishnavi2007} and \cite{Saunders2009} is assumed for this thesis.
% TODO: check for Vaishnavi reference
The perception of multidimensional data that is represented in virtual reality can be different for every social actor. There is no objective truth in how the visualization is interpreted, it is purely subjective. In addition, the assumed possibilities to interact with the visualization in virtual reality are also depending on the interpretation of each individual. Finally, due to the application of design science as the research strategy, the researcher also is part of what is researched and thus is subjective.


%----------------------------------------------------------------------------------------
%	SECTION 3
%----------------------------------------------------------------------------------------

\section{Approach}

blub


%----------------------------------------------------------------------------------------
%	SECTION 4
%----------------------------------------------------------------------------------------

\section{Strategy}

blub


%----------------------------------------------------------------------------------------
%	SECTION 5
%----------------------------------------------------------------------------------------

\section{Timeline}

blub


%----------------------------------------------------------------------------------------
%	SECTION 6
%----------------------------------------------------------------------------------------

% \section{Data collecton}


%----------------------------------------------------------------------------------------
%	SECTION 7
%----------------------------------------------------------------------------------------

% \section{Data analysis}


%----------------------------------------------------------------------------------------
%	SECTION 8
%----------------------------------------------------------------------------------------

\section{Conclusion}


